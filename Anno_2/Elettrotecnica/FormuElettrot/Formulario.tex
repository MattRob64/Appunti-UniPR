\documentclass{article}
\usepackage{graphicx} % Required for inserting images

\title{Formulario Elettrotecnica}
\author{Mattia Robuschi Caprara}
\date{}

\begin{document}

\maketitle

\newpage

\tableofcontents

\newpage

\section{Carica elettrone}
La carica di un elettrone ($e^-$) è: $-1.6 \cdot 10^{-19} C$ \\
L'unita di misura utilizzata è il Coulomb.

\section{Prima legge di Ohm}
$\Delta V = R \cdot I \;\; [V]$ \\ \\
$\Delta V$ è la differenza di potenziale che si misura in Volt $[V]$ \\
$R$ è la resistenza che si misura in Ohm $[\Omega]$ \\
$I$ è la corrente elettrica che si misura in Ampere $[A]$

\section{Seconda legge di Ohm}
$R = \varphi \cdot \frac{l}{S} \;\; [\Omega \cdot m]$ \\ \\
$R$ indica la resistenza \\
$\varphi$ indica la resistività, ovvero quanto un materiale resiste all'elettricità \\
$l$ indica la lunghezza del campione in cui scorre la corrente \\
$S$indica la sezione/area del campione in cui scorre la corrente \\ \\
L'inverso della resistività è la conducibilità, ovvero quanto un materiale conduce elettricità, viene indicata con:
\[ \delta = \frac{1}{\varphi} \]

\end{document}