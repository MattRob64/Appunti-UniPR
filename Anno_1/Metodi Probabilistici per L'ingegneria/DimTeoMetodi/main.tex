\documentclass{article}
\usepackage{graphicx} % Required for inserting images

\title{Dimostrazioni e Domande di Teoria Metodi Probabilistici}
\author{Mattia Robuschi Caprara}
\date{}

\begin{document}

\maketitle

\newpage

\tableofcontents

\newpage

\section{Domande di Teoria}
\subsection{}
\section{Dimostrazioni e Concetti di Teoria}
\subsection{Formula del doppio condizionamento}
(8.11) libro
\subsection{Teorema Fondamentale con g(x) continua}
Pagina 50
\subsection{Distinzione fra Trasformazioni di V.A e Vettori Aleatori}
Pagina 99
\subsection{Seconda proprietà CDF congiunta}
Pagina 100-101
\subsection{Sesta proprietà CDF congiunta}
Pagina 101-102
\subsection{Settima proprietà CDF congiunta}
Pagina 102-103
\subsection{Calcolo CDF,PDF e PMF marginali}
Pagina non indicata
\subsection{PDF congiunta $\geq 0$}
Pagina 106-107
\subsection{Calcolare probabilit`a che due valori appartengano ad un dominio tramite PDF congiunta}
Pagina 107
\subsection{Dimostrazione teorema di Bayes CDF condizionata ad una V.A}
Pagina 109
\subsection{Calcolare probabilità con PDF condizionata}
Pagina 112
\subsection{Proprietà 1 Trasformazioni V.A indip.}
Pagina 112
\subsection{Proprietà 3 Trasformazioni V.A indip.}
Pagina 113
\subsection{Proprietà 5 Trasformazioni V.A indip.}
Pagina 113
\subsection{Trasformazione di 1 V.A Date 2 V.A caso discreto}
Pagina 114
\subsection{Trasformazione di 1 V.A Date 2 V.A caso continuo}
Pagina 115
\subsection{Trasformazione di 2 V.A Date 2 V.A caso discreto}
Pagina 116
\subsection{Trasformazione di 2 V.A Date 2 V.A caso contnuo metodo grafico}
Pagina 117
\subsection{Trasformazione di 2 V.A Date 2 V.A caso contnuo teorema fondamentale}
Pagina 117
\subsection{Metodo alternativo per il calcolo del determinante della matrice Jacobiana}
Pagina 117
\subsection{Teorema fondamentale caso 1 V.A da trasformare}
Pagina 117
\subsection{Utilizzo coordinate polari}
Pagina 121
\subsection{Calcolo valor medio di 2 V.A indipendenti}
Pagina 123
\subsection{Calcolo varianza con 2 V.A indipendenti}
Pagina 123
\subsection{Teorema della media condizionata}
Pagina 123-124
\subsection{Momenti ordinari congiunti di ordine 1}
Pagina 124
\subsection{Correlazione}
Pagina 124
\subsection{Momenti ordinari congiunti di ordine 1}
Pagina 125
\subsection{Covarianza}
Pagina 125
\subsection{Momenti centrali noti}
Pagina 125
\subsection{Incorrelazione e Ortogonalità}
Pagina 125
\subsection{Definire covarianza tramite correlazione utilizzando lo scarto}
Pagina 125
\subsection{Relazione fra correlazione e covarianza}
Pagina 125
\subsection{Proprietà 1 correlazione per V.A indipendenti}
Pagina 125-126
\subsection{Proprietà 2 correlazione per V.A indipendenti}
Pagina 126
\subsection{Proprietà 4 correlazione per V.A indipendenti (disuguaglianza correlazione)}
Pagina 126-127
\subsection{Proprietà 4 applicata alla covarianza}
Pagina 127
\subsection{Coefficiente di correlazione}
Pagina 127
\subsection{V.A congiuntamente gaussiane ed incorrelate, implica indipendenza}
PDF prof.
\end{document}
